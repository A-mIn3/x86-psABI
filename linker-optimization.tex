\chapter{Linker Optimization}

This chapter describes optimizations which may be performed by linker.

\section{Combine GOTPLT and GOT Slots}
In the small and medium models, when there are both PLT and GOT references
to the same function symbol, normally linker creates a GOTPLT slot for PLT
entry and a GOT slot for GOT reference.  A run-time JUMP_SLOT relocation is
created to update the GOTPLT slot and a run-time GLOB_DAT relocation is
created to update the GOT slot.  Both JUMP_SLOT and GLOB_DAT relocations
apply the same symbol value to GOTPLT and GOT slots, respectively, at
run-time.

As an optimization, linker may combine GOTPLT and GOT slots into a single
GOT slot and remove the run-time JUMP_SLOT relocation.  It replaces the
regular PLT entry:

\begin{figure}[H]
\Hrule
\caption{Procedure Linkage Table Entry Via GOTPLT Slot}
\label{gotplt_plt}
\begin{footnotesize}
\begin{verbatim}
  .PLT: jmp      [GOTPLT slot]
        pushl    relocation index
        jmp      .PLT0
\end{verbatim}
\end{footnotesize}
\Hrule
\end{figure}

\noindent
with an GOT PLT entry with an indirect jump via the GOT slot:
\indent

\begin{figure}[H]
\Hrule
\caption{Procedure Linkage Table Entry Via GOT Slot}
\label{got_plt}
\begin{footnotesize}
\begin{verbatim}
  .PLT: jmp      [GOT slot]
        nop
\end{verbatim}
\end{footnotesize}
\Hrule
\end{figure}

\noindent
and resolves the PLT reference to the GOT PLT entry.  Indirect \code{jmp}
is an 5-byte instruction.  \code{nop} can be encoded as a 3-byte
instruction or a 11-byte instruction for 8-byte or 16-byte PLT slot.
A separate PLT with 8-byte slots may be used for this optimization.
\indent

This optimization isn't applicable to the \texttt{STT_GNU_IFUNC} symbols
since their GOTPLT slots are resolved to the selected implementation and
their GOT slots are resolved to their PLT entries.

This optimization must be avoided if pointer equality is needed since
the symbol value won't be cleared in this case and the dynamic linker
won't update the GOT slot.  Otherwise, the resulting binary will get
into an infinite loop at run-time.

\section{Optimize R_386_GOT32X Relocation}

The \xARCH instruction encoding supports converting certain instructions
on memory operand with \texttt{R_386_GOT32X} relocation against symbol,
\texttt{foo}, into a different form on immediate operand if \texttt{foo}
is defined locally.

\begin{description}
\item[\textindex{Convert call, jmp and mov}]
  Convert memory operand of \code{call}, \code{jmp} and \code{mov} into
  immediate operand.

\begin{table}[H]
\Hrule
\caption{Call, Jmp and Mov Conversion}
\begin{center}
\small\code{
\begin{tabular}{l|l}
\multicolumn{1}{c}{Memory Operand} & \multicolumn{1}{c}{Immediate Operand} \\
\hline
call *foo@GOT(\%reg) & nop call foo \\
call *foo@GOT(\%reg) & call foo nop \\
jmp  *foo@GOT(\%reg) & jmp foo nop \\
mov  foo@GOT(\%reg1), \%reg2 & lea foo@GOTOFF(\%reg1), \%reg2 \\
\end{tabular}
}
\end{center}
\Hrule
\end{table}

\item[\textindex{Convert Test and Binop}]
  Convert memory operand of \code{call}, \code{jmp}, \code{mov},
  \code{test} and \code{binop} into immediate operand, where \code{binop}
  is one of \code{adc}, \code{add}, \code{and}, \code{cmp}, \code{or},
  \code{sbb}, \code{sub}, \code{xor} instructions, when
  position-independent code is disabled.

\begin{table}[H]
\Hrule
\caption{Test and Binop Conversion}
\begin{center}
\small\code{
\begin{tabular}{l|l}
\multicolumn{1}{c}{Memory Operand} & \multicolumn{1}{c}{Immediate Operand} \\
\hline
call *foo@GOT & nop call foo \\
call *foo@GOT & call foo nop \\
jmp  *foo@GOT & jmp foo nop \\
mov  foo@GOT, \%reg & mov \$foo, \%reg \\
test \%reg, foo@GOT & test \$foo, \%reg \\
binop foo@GOT, \%reg & binop \$foo, \%reg \\
call *foo@GOT(\%reg) & nop call foo \\
call *foo@GOT(\%reg) & call foo nop \\
jmp  *foo@GOT(\%reg) & jmp foo nop \\
mov foo@GOT(\%reg1), \%reg2 & mov \$foo, \%reg2 \\
test  \%reg1, name@GOT(\%reg2) & test \$foo, \%reg1 \\
binop name@GOT(\%reg1), \%reg2 & binop \$foo, \%reg2 \\
\end{tabular}
}
\end{center}
\Hrule
\end{table}

\end{description}
