
\chapter{Object Files}

\section{Sections}

\subsection{Special Sections}

\begin{table}[H]
\Hrule
  \caption{Special sections}
  \begin{center}
    \begin{tabular}[t]{l|l|l}
      \multicolumn{1}{c}{Name} & \multicolumn{1}{c}{Type}
       & \multicolumn{1}{c}{Attributes} \\
      \hline
      \texttt{.eh_frame} & \texttt{SHT_PROGBITS} & \texttt{SHF_ALLOC}
    \end{tabular}
  \end{center}
\Hrule
\end{table}

\begin{description}
 \item[.eh_frame] This section holds the unwind function table.
                      The contents are described in Section~\ref{sec_eh_frame}
                      of this document.
\end{description}

\subsection{EH\_FRAME sections}
\label{sec_eh_frame}

The call frame information needed for unwinding the stack is output into
one section named
\code{.eh_frame}.  An \code{.eh_frame} section consists of one or more
subsections. Each subsection contains a CIE (Common Information Entry)
followed by varying number of FDEs (Frame Descriptor Entry). A FDE
corresponds to an explicit or compiler generated function in a
compilation unit, all FDEs can access the CIE that begins their
subsection for data.  If the code for a function is not one contiguous
block, there will be a separate FDE for each contiguous sub-piece.

If an object file contains C++ template instantiations there shall be
a separate CIE immediately preceding each FDE corresponding to an
instantiation.

Using the preferred encoding specified below, the \code{.eh_frame} section can
be entirely resolved at link time and thus can become part of the
text segment.

\code{EH_PE} encoding below refers to the pointer encoding as specified in the
enhanced LSB Chapter 7 for \code{Eh_Frame_Hdr}.

\begin{table}[H]
\Hrule
\caption{Common Information Entry (CIE)}
\label{format-cie}
\begin{center}
\begin{tabular}{p{7em}|l|p{17em}}
  \multicolumn{1}{c}{Field}
         & \multicolumn{1}{c}{Length (byte)}
         & \multicolumn{1}{c}{Description} \\ \hline
  Length & 4 & Length of the CIE (not including this 4-byte field) \\
  CIE id & 4 & Value 0 for \code{.eh_frame} (used to distinguish CIEs and
		  FDEs when scanning the section) \\
  Version & 1 & Value One (1) \\
  CIE Augmentation String & string & Null-terminated string with legal
     values being "" or 'z' optionally followed by single occurrances of
     'P', 'L', or 'R' in any order.  The presence of character(s) in the
     string dictates the content of field 8, the Augmentation Section.  Each
     character has one or two associated operands in the AS (see
     table~\ref{format-cieaug} for which ones).  Operand order
     depends on position in the string ('z' must be first). \\
  Code Align Factor & uleb128 & To be multiplied with the
    "Advance Location" instructions in the Call Frame Instructions \\
  Data Align Factor & sleb128 & To be multiplied with all offsets
                                in the Call Frame Instructions \\
  Ret Address Reg & 1/uleb128 &  A "virtual" register representation
                                 of the return address. In Dwarf V2,
			         this is a byte, otherwise it is
			         uleb128. It is a byte in gcc 3.3.x \\
  Optional CIE Augmentation Section & varying & Present if Augmentation
                     String in Augmentation Section field 4 is not 0. 
		     See table~\ref{format-cieaug} for the content. \\
  Optional Call Frame Instructions & varying & \\
\hline
    \end{tabular}
  \end{center}
\Hrule
\end{table}

\begin{table}[H]
\Hrule
\caption{CIE Augmentation Section Content}
\label{format-cieaug}
\begin{center}
\begin{tabular}{l|p{6em}|l|p{16em}}
  \multicolumn{1}{c}{Char}
         & \multicolumn{1}{c}{Operands}
         & \multicolumn{1}{c}{Length (byte)}
         & \multicolumn{1}{c}{Description} \\ \hline
  z & size & uleb128 & Length of the remainder of the
                                        Augmentation Section \\
  P & personality_enc & 1 & Encoding specifier - preferred
                            value is a pc-relative, signed
                            4-byte \\
    & personality routine & (encoded) & Encoded pointer to personality
                                        routine (actually to the PLT
                                        entry for the personality
                                        routine) \\
  R & code_enc & 1 & Non-default encoding for the
                     code-pointers (FDE members
                     \code{initial_location} and \code{address_range}
                     and the operand for \code{DW_CFA_set_loc})
                     - preferred value is pc-relative,
                     signed 4-byte \\
  L & lsda_enc & 1 & FDE augmentation bodies may contain
		     LSDA pointers. If so they are encoded
		     as specified here -
		     preferred value is pc-relative, signed 4-byte possibly
		     indirect thru a GOT entry \\
\hline
    \end{tabular}
  \end{center}
\Hrule
\end{table}

\begin{table}[H]
\Hrule
\caption{Frame Descriptor Entry (FDE)}
\label{format-fde}
\begin{center}
\begin{tabular}{p{7em}|l|p{17em}}
  \multicolumn{1}{c}{Field}
         & \multicolumn{1}{c}{Length (byte)}
         & \multicolumn{1}{c}{Description} \\ \hline
  Length & 4 & Length of the FDE (not including this 4-byte field) \\
  CIE pointer & 4 & Distance from this field to the
		    nearest preceding CIE (the value is subtracted from the
		    current address). This value can never be zero and thus can
		    be used to distinguish CIE's and FDE's when scanning the
		    \code{.eh_frame} section \\
  Initial Location & var & Reference to the function code 
                           corresponding to this FDE.
                           If 'R' is missing from the CIE
                           Augmentation String, the field is an
                           8-byte absolute pointer. Otherwise,
                           the corresponding \code{EH_PE} encoding in the 
                           CIE Augmentation Section is used to 
                           interpret the reference \\
  Address Range & var & Size of the function code corresponding
                       to this FDE.
                       If 'R' is missing from the CIE
                       Augmentation String, the field is an
                       8-byte unsigned number. Otherwise,
                       the size is determined by the
                       corresponding \code{EH_PE} encoding in the 
                       CIE Augmentation Section (the
                       value is always absolute) \\
  Optional FDE Augmentation Section & var & Present if CIE Augmentation
                     String is non-empty.
		     See table~\ref{format-fdeaug} for the content. \\
  Optional Call Frame Instructions & var & \\
\hline
    \end{tabular}
  \end{center}
\Hrule
\end{table}

\begin{table}[H]
\Hrule
\caption{FDE Augmentation Section Content}
\label{format-fdeaug}
\begin{center}
\begin{tabular}{l|p{6em}|l|p{16em}}
  \multicolumn{1}{c}{Char}
         & \multicolumn{1}{c}{Operands}
         & \multicolumn{1}{c}{Length (byte)}
         & \multicolumn{1}{c}{Description} \\ \hline
  z & length & uleb128 & Length of the remainder of the
                                        Augmentation Section \\
  L & LSDA & var & LSDA pointer, encoded in the
                   format specified by the
                   corresponding operand in the CIE's
                   augmentation body. (only present if length > 0). \\
\hline
    \end{tabular}
  \end{center}
\Hrule
\end{table}
The existence and size of the optional call frame instruction area must
be computed
based on the overall size and the offset reached while scanning the
preceding fields of the CIE or FDE.

The overall size of a \code{.eh_frame} section is given in the ELF section
header.  The only way to determine the number of entries is to scan
the section until the end, counting entries as they are encountered.

\section{Symbol Table}

The discussion of "Function Addresses" in Section~\ref{function_addresses}
defines some special values for symbol table fields.

The \texttt{STT_GNU_IFUNC}
\footnote{It is specified in {\bf ifunc.txt}
at \url{http://sites.google.com/site/x32abi/documents}}
symbol type is optional. It is the same as
\texttt{STT_FUNC} except that it always points to a function or piece of
executable code which takes no arguments and returns a function pointer.
If an \texttt{STT_GNU_IFUNC} symbol is referred to by a relocation, then
evaluation of that relocation is delayed until load-time.  The value
used in the relocation is the function pointer returned by an invocation
of the \texttt{STT_GNU_IFUNC} symbol.
 
The purpose of the \texttt{STT_GNU_IFUNC} symbol type is to allow the
run-time to select between multiple versions of the implementation of
a specific function.  The selection made in general will take the
currently available hardware into account and select the most
appropriate version.

\section{Relocation}

\subsection{Relocation Types}

Figure~\ref{reloc_fields} shows the allowed relocatable fields.

\begin{figure}[H]
\label{reloc_fields}
\Hrule
  \caption{Relocatable Fields}
\begin{center}
  \begin{picture}(360,190)
    \put(0,150){\framebox(50, 33){7\hfill\textit{word8}\hfill 0}}
    \put(0,100){\framebox(90, 33){15\hfill\textit{word16}\hfill 0}}
    \put(0,50){\framebox(180, 33){31\hfill\textit{word32}\hfill 0}}
  \end{picture}
\end{center}
\Hrule
\end{figure}

\noindent
\begin{tabular*}{\textwidth}{l@{\extracolsep{\fill}}p{4in}}
\textit{word8} & This specifies a 8-bit field occupying 1 byte.\\
\textit{word16} & This specifies a 16-bit field occupying 2 bytes
                  with arbitrary byte alignment.  These values use
                  the same byte order as other word values in the
                  \xARCH architecture. \\
\textit{word32} & This specifies a 32-bit field occupying 4 bytes
                  with arbitrary byte alignment.  These values use
                  the same byte order as other word values in the
                  \xARCH architecture. \\
\end{tabular*}

The following notations are used for specifying relocations in table~
\ref{tab-relocations}:
\begin{description}
\item[A] Represents the addend used to compute the value of the
  relocatable field.
\item[B] Represents the base address at which a shared object has been
  loaded into memory during execution.  Generally, a shared object is
  built with a 0 base virtual address, but the execution address will
  be different.
\item[G] Represents the offset into the global offset table at which 
  the relocation entry's symbol will reside during execution.
\item[GOT] Represents the address of the global offset table.
\item[L] Represents the place (section offset or address) of the
  \textindex{Procedure Linkage Table} entry for a symbol.
\item[P] Represents the place (section offset or address) of the
  storage unit being relocated (computed using \code{r_offset}).
\item[S] Represents the value of the symbol whose index resides in the
  relocation entry.
\item[Z] Represents the size of the symbol whose index resides in the
  relocation entry.
\end{description}

\begin{table}[H]
\Hrule
  \caption{Relocation Types}
  \small
  \label{tab-relocations}
  \begin{center}
    \myfontsize
    \begin{tabular}[t]{l|r|l|l}
      \multicolumn{1}{c}{Name} & 
      \multicolumn{1}{c}{Value} & 
      \multicolumn{1}{c}{Field} & 
      \multicolumn{1}{c}{Calculation} \\
      \hline
      \texttt{R_386_NONE}  & 0 & none & none \\
      \texttt{R_386_32}    & 1 & \textit{word64} & \texttt{S + A} \\
      \texttt{R_386_PC32}  & 2 & \textit{word32} & \texttt{S + A - P} \\
      \texttt{R_386_GOT32} & 3 & \textit{word32} & \texttt{G + A} \\
      \texttt{R_386_PLT32} & 4 & \textit{word32} & \texttt{L + A - P} \\
      \texttt{R_386_COPY}  & 5 & none            & none \\
      \texttt{R_386_GLOB_DAT} & 6 & \textit{wordclass} & \texttt{S} \\
      \texttt{R_386_JUMP_SLOT} & 7 & \textit{wordclass} & \texttt{S} \\
      \texttt{R_386_RELATIVE} & 8 & \textit{wordclass} & \texttt{B + A} \\
      \texttt{R_386_GOTOFF} $^\dagger$ & 9 & \textit{word64} & \texttt{S + A - GOT} \\
      \texttt{R_386_GOTPC} & 10 & \textit{word32} & \texttt{GOT + A - P} \\
      \texttt{R_386_TLS_TPOFF} & 14 & \textit{word32} &  \\
      \texttt{R_386_TLS_IE} & 15 & \textit{word32} &  \\
      \texttt{R_386_TLS_GOTIE} & 16 & \textit{word32} &  \\
      \texttt{R_386_TLS_LE} & 17 & \textit{word32} &  \\
      \texttt{R_386_TLS_GD} & 18 & \textit{word32} &  \\
      \texttt{R_386_TLS_LDM} & 19 & \textit{word32} &  \\
      \texttt{R_386_16}    & 20 & \textit{word16} & \texttt{S + A} \\
      \texttt{R_386_PC16}  & 21 & \textit{word16} & \texttt{S + A - P} \\
      \texttt{R_386_8}     & 22 & \textit{word8} & \texttt{S + A} \\
      \texttt{R_386_PC8}   & 23 & \textit{word8} & \texttt{S + A - P} \\
      \texttt{R_386_TLS_GD_32} & 24 & \textit{word32} &  \\
      \texttt{R_386_TLS_GD_PUSH} & 25 & \textit{word32} &  \\
      \texttt{R_386_TLS_GD_CALL} & 26 & \textit{word32} &  \\
      \texttt{R_386_TLS_GD_POP} & 27 & \textit{word32} &  \\
      \texttt{R_386_TLS_LDM_32} & 28 & \textit{word32} &  \\
      \texttt{R_386_TLS_LDM_PUSH} & 29 & \textit{word32} &  \\
      \texttt{R_386_TLS_LDM_CALL} & 30 & \textit{word32} &  \\
      \texttt{R_386_TLS_LDM_POP} & 31 & \textit{word32} &  \\
      \texttt{R_386_TLS_LDO_32} & 32 & \textit{word32} &  \\
      \texttt{R_386_TLS_IE_32} & 33 & \textit{word32} &  \\
      \texttt{R_386_TLS_LE_32} & 34 & \textit{word32} &  \\
      \texttt{R_386_TLS_DTPMOD32} & 35 & \textit{word32} &  \\
      \texttt{R_386_TLS_DTPOFF32} & 36 & \textit{word32} &  \\
      \texttt{R_386_TLS_TPOFF32} & 37 & \textit{word32} &  \\
      \texttt{R_386_SIZE32} & 38 & \textit{word32} & \texttt{Z + A} \\
      \texttt{R_386_TLS_GOTDESC} & 39 & \textit{word32} &  \\
      \texttt{R_386_TLS_DESC_CALL} & 40 & none & none \\
      \texttt{R_386_TLS_DESC} & 41 & \textit{word32} &  \\
      \texttt{R_386_IRELATIVE} & 42 & \textit{wordclass} & \texttt{indirect (B + A)}\\
     \cline{1-4}
    \end{tabular}
  \end{center}
\Hrule
\end{table}

\begin{sloppypar}
A program or object file using \texttt{R_386_8},
\texttt{R_386_16}, \texttt{R_386_PC16} or \texttt{R_386_PC8}
relocations is not conformant to this ABI, these relocations are only
added for documentation purposes.  The \texttt{R_386_16}, and
\texttt{R_386_8} relocations truncate the computed value to 16-bits
and 8-bits respectively.
\end{sloppypar}

\begin{sloppypar}
The relocations \texttt{R_386_TLS_TPOFF},
\texttt{R_386_TLS_IE}, \texttt{R_386_TLS_GOTIE},
\texttt{R_386_TLS_LE}, \texttt{R_386_TLS_GD},
\texttt{R_386_TLS_LDM}, \texttt{R_386_TLS_GD_32},
\texttt{R_386_TLS_GD_PUSH}, \texttt{R_386_TLS_GD_CALL},
\texttt{R_386_TLS_GD_POP}, \texttt{R_386_TLS_LDM_32},
\texttt{R_386_TLS_LDM_PUSH}, \texttt{R_386_TLS_LDM_CALL},
\texttt{R_386_TLS_LDM_POP}, \texttt{R_386_TLS_LDO_32},
\texttt{R_386_TLS_IE_32}, \texttt{R_386_TLS_LE_32},
\texttt{R_386_TLS_DTPMOD32}, \texttt{R_386_TLS_DTPOFF32} and
\texttt{R_386_TLS_TPOFF32} are listed for completeness.  They are part
of the Thread-Local Storage ABI extensions and are documented in the
document called ``ELF Handling for Thread-Local
Storage''\footnote{This document is currently available via
  \url{http://people.redhat.com/drepper/tls.pdf}}\index{Thread-Local
  Storage}.  The relocations \texttt{R_386_TLS_GOTDESC},
\texttt{R_386_TLS_DESC_CALL} and \texttt{R_386_TLS_DESC} are also
used for Thread-Local Storage, but are not documented there as of this
writing.  A description can be found in the document ``Thread-Local
Storage Descriptors for IA32 and AMD64/EM64T''\footnote{This document
  is currently available via
  \url{http://people.redhat.com/aoliva/writeups/TLS/RFC-TLSDESC-x86.txt}}.
\end{sloppypar}

\texttt{R_386_IRELATIVE} is similar to \texttt{R_386_RELATIVE}
except that the value used in this relocation is the program address
returned by the function, which takes no arguments, at the address of
the result of the corresponding \texttt{R_386_RELATIVE} relocation.

One use of the \texttt{R_386_IRELATIVE} relocation is to avoid name
lookup for the locally defined \texttt{STT_GNU_IFUNC} symbols at
load-time.  Support for this relocation is optional, but is required for
the \texttt{STT_GNU_IFUNC} symbols.


%%% Local Variables: 
%%% mode: latex
%%% TeX-master: "abi"
%%% End: 
